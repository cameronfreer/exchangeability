% Blueprint content for the Exchangeability formalization project
% Structure follows Kallenberg (2005), Chapter 1, Theorem 1.1

\chapter{Introduction}

This blueprint documents the formalization of \textbf{de Finetti's theorem} and the
\textbf{de Finetti--Ryll-Nardzewski equivalence} for infinite sequences on \emph{standard Borel spaces}.

The main result establishes a three-way equivalence between:
\begin{itemize}
  \item \textbf{Contractable}: All strictly increasing subsequences of equal length have the same distribution
  \item \textbf{Exchangeable}: Distribution invariant under finite permutations
  \item \textbf{Conditionally i.i.d.}: There exists a probability kernel such that finite marginals equal mixtures of product measures
\end{itemize}

We formalize \emph{all three proofs} from Kallenberg (2005):
\begin{enumerate}
  \item \textbf{Koopman/Ergodic approach} using the Mean Ergodic Theorem
  \item \textbf{L$^2$ approach} using elementary contractability bounds
  \item \textbf{Martingale approach} using reverse martingale convergence (after Aldous)
\end{enumerate}


\chapter{Foundations}

\section{Core Definitions}

\begin{definition}[Exchangeable sequence]
  \label{def:exchangeable}
  \lean{Exchangeability.Exchangeable}
  \leanok
  A sequence $(X_n)_{n \in \mathbb{N}}$ of random variables is \emph{exchangeable}
  with respect to measure $\mu$ if for every $n \in \mathbb{N}$ and every permutation
  $\sigma$ of $\{0, \ldots, n-1\}$, the joint distribution of $(X_{\sigma(0)}, \ldots, X_{\sigma(n-1)})$
  equals that of $(X_0, \ldots, X_{n-1})$.
\end{definition}

\begin{definition}[Contractable sequence]
  \label{def:contractable}
  \lean{Exchangeability.Contractable}
  \leanok
  \uses{def:exchangeable}
  A sequence $(X_n)_{n \in \mathbb{N}}$ is \emph{contractable} with respect to measure $\mu$
  if for all $m \in \mathbb{N}$ and all strictly increasing functions $k, k' : \mathrm{Fin}(m) \to \mathbb{N}$,
  the joint distribution of $(X_{k(0)}, \ldots, X_{k(m-1)})$ equals that of
  $(X_{k'(0)}, \ldots, X_{k'(m-1)})$.
\end{definition}

\begin{definition}[Conditionally i.i.d.\ sequence]
  \label{def:conditionallyIID}
  \lean{Exchangeability.ConditionallyIID}
  \leanok
  A sequence $(X_n)_{n \in \mathbb{N}}$ is \emph{conditionally i.i.d.}\ with respect to
  measure $\mu$ if there exists a probability kernel $\nu : \Omega \to \mathrm{Measure}(\alpha)$
  such that for any strictly monotone $k : \mathrm{Fin}(m) \to \mathbb{N}$, the joint distribution
  of $(X_{k(0)}, \ldots, X_{k(m-1)})$ equals the mixture $\mu.\mathrm{bind}(\omega \mapsto \nu(\omega)^{\otimes m})$.
\end{definition}

\section{\texorpdfstring{$\sigma$}{σ}-algebra Infrastructure}

\begin{definition}[Tail $\sigma$-algebra]
  \label{def:tailSigma}
  \lean{Exchangeability.DeFinetti.ViaMartingale.tailSigma}
  \leanok
  The \emph{tail $\sigma$-algebra} of a sequence $(X_n)$ is
  $\mathcal{T} = \bigcap_{n=0}^{\infty} \sigma(X_n, X_{n+1}, \ldots)$.
\end{definition}

\begin{definition}[Future filtration]
  \label{def:futureFiltration}
  \lean{Exchangeability.DeFinetti.ViaMartingale.futureFiltration}
  \leanok
  The \emph{future filtration} at level $m$ is $\mathcal{F}_m = \sigma(X_{m+1}, X_{m+2}, \ldots)$.
\end{definition}

\begin{lemma}[Future filtration is antitone]
  \label{lem:futureFiltration_antitone}
  \lean{Exchangeability.DeFinetti.ViaMartingale.futureFiltration_antitone}
  \leanok
  \uses{def:futureFiltration}
  The future filtration is antitone: $m \le n$ implies $\mathcal{F}_n \le \mathcal{F}_m$.
\end{lemma}

\begin{lemma}[Tail $\sigma$-algebra contained in future filtration]
  \label{lem:tailSigma_le_futureFiltration}
  \lean{Exchangeability.DeFinetti.ViaMartingale.tailSigma_le_futureFiltration}
  \leanok
  \uses{def:tailSigma, def:futureFiltration}
  For all $m$: $\mathcal{T} \le \mathcal{F}_m$.
\end{lemma}

\begin{lemma}[Tail $\sigma$-algebra is infimum of reverse filtration]
  \label{lem:tailSigma_eq_iInf}
  \lean{Exchangeability.DeFinetti.ViaMartingale.tailSigma_eq_iInf_rev}
  \leanok
  \uses{def:tailSigma, def:futureFiltration}
  $\mathcal{T} = \bigwedge_{m=0}^{\infty} \mathcal{F}_m$.
\end{lemma}

\begin{definition}[Shift-invariant $\sigma$-algebra]
  \label{def:shiftInvariantSigma}
  \lean{Exchangeability.DeFinetti.ViaKoopman.shiftInvariantSigma}
  \leanok
  The shift-invariant $\sigma$-algebra consists of sets $S$ such that
  $\theta^{-1}(S) = S$ where $\theta$ is the shift operator.
\end{definition}


\chapter{Easy Directions}

\section{Exchangeable implies Contractable}

\begin{lemma}[Permutation extension]
  \label{lem:perm_extension}
  \lean{Exchangeability.exists_perm_extending_strictMono}
  \leanok
  Any strictly increasing function $k : \mathrm{Fin}(m) \to \mathbb{N}$ with range
  contained in $\{0, \ldots, n-1\}$ extends to a permutation of $\{0, \ldots, n-1\}$.
\end{lemma}

\begin{theorem}[Exchangeable implies Contractable]
  \label{thm:contractable_of_exchangeable}
  \lean{Exchangeability.contractable_of_exchangeable}
  \leanok
  \uses{def:exchangeable, def:contractable, lem:perm_extension}
  If $(X_n)$ is exchangeable, then it is contractable.
\end{theorem}

\section{Conditionally i.i.d.\ implies Exchangeable}

\begin{theorem}[Conditionally i.i.d.\ implies Exchangeable]
  \label{thm:exchangeable_of_conditionallyIID}
  \lean{Exchangeability.exchangeable_of_conditionallyIID}
  \leanok
  \uses{def:conditionallyIID, def:exchangeable}
  If $(X_n)$ is conditionally i.i.d., then it is exchangeable.
\end{theorem}


\chapter{Main Implication: Contractable implies Conditionally i.i.d.}

This is the deep direction of de Finetti's theorem. We formalize three independent proofs.

\section{Via Martingale (Aldous' proof)}

The martingale approach uses reverse martingale convergence to the tail $\sigma$-algebra.

\subsection{Pair Law Equality}

\begin{lemma}[Pair law equality for contractable sequences]
  \label{lem:pair_law_eq}
  \lean{Exchangeability.DeFinetti.ViaMartingale.pair_law_eq_of_contractable}
  \leanok
  \uses{def:contractable}
  For a contractable sequence and $k \le m$, the joint distribution of
  $(X_k, X_{m+1}, X_{m+2}, \ldots)$ equals that of $(X_m, X_{m+1}, X_{m+2}, \ldots)$.
\end{lemma}

\begin{lemma}[Contractable distribution equality]
  \label{lem:contractable_dist_eq}
  \lean{Exchangeability.DeFinetti.ViaMartingale.contractable_dist_eq}
  \leanok
  \uses{def:contractable, def:futureFiltration}
  For a contractable sequence, the joint distribution of $(X_k, \theta_{m+1} X)$ equals
  that of $(X_m, \theta_{m+1} X)$ for all $k \le m$, where $\theta_n$ is the shift operator.
\end{lemma}

\begin{lemma}[Conditional expectation of indicator equals under contractability]
  \label{lem:condExp_indicator_eq}
  \lean{Exchangeability.DeFinetti.ViaMartingale.condExp_indicator_eq_of_contractable}
  \leanok
  \uses{lem:contractable_dist_eq, def:futureFiltration}
  For contractable sequences and $k \le m$:
  $\mathbb{E}[\mathbf{1}_{X_k \in B} \mid \mathcal{F}_m] = \mathbb{E}[\mathbf{1}_{X_m \in B} \mid \mathcal{F}_m]$ a.s.
\end{lemma}

\subsection{Kallenberg Chain and Convergence}

\begin{lemma}[Kallenberg chain lemma]
  \label{lem:kallenberg_chain}
  \lean{Exchangeability.DeFinetti.ViaMartingale.condExp_indicator_revFiltration_eq_tail}
  \leanok
  \uses{lem:condExp_indicator_eq, def:tailSigma, lem:tailSigma_eq_iInf, lem:futureFiltration_antitone}
  Conditional expectations of indicators given the reverse filtration
  converge to conditional expectations given the tail $\sigma$-algebra.
\end{lemma}

\begin{lemma}[Conditional expectation convergence]
  \label{lem:condexp_convergence}
  \lean{Exchangeability.DeFinetti.ViaMartingale.condexp_convergence}
  \leanok
  \uses{lem:pair_law_eq}
  For $k \le m$ and measurable $B$:
  $\mathbb{E}[\mathbf{1}_{X_m \in B} \mid \mathcal{F}_m] = \mathbb{E}[\mathbf{1}_{X_k \in B} \mid \mathcal{F}_m]$ a.s.
\end{lemma}

\begin{lemma}[Extreme members equal on tail]
  \label{lem:extreme_members_equal}
  \lean{Exchangeability.DeFinetti.ViaMartingale.extreme_members_equal_on_tail}
  \leanok
  \uses{lem:condexp_convergence, lem:kallenberg_chain}
  For any measurable $B$:
  $\mathbb{E}[\mathbf{1}_{X_m \in B} \mid \mathcal{T}] = \mathbb{E}[\mathbf{1}_{X_0 \in B} \mid \mathcal{T}]$ a.s.
\end{lemma}

\subsection{Factorization and Directing Measure}

\begin{lemma}[Finite level factorization]
  \label{lem:finite_level_factorization}
  \lean{Exchangeability.DeFinetti.ViaMartingale.finite_level_factorization}
  \leanok
  \uses{lem:extreme_members_equal, def:futureFiltration}
  For finite products of indicators, the conditional expectation given $\mathcal{F}_m$
  factors as a product of individual conditional expectations.
\end{lemma}

\begin{lemma}[Tail factorization from future]
  \label{lem:tail_factorization}
  \lean{Exchangeability.DeFinetti.ViaMartingale.tail_factorization_from_future}
  \leanok
  \uses{lem:finite_level_factorization, def:tailSigma, lem:tailSigma_le_futureFiltration}
  The tail $\sigma$-algebra factorization follows from the finite-level factorization
  via reverse martingale convergence.
\end{lemma}

\begin{lemma}[Block coordinate conditional independence]
  \label{lem:block_coord_condIndep}
  \lean{Exchangeability.DeFinetti.ViaMartingale.block_coord_condIndep}
  \leanok
  \uses{lem:tail_factorization, def:tailSigma}
  Coordinates are conditionally independent given the tail $\sigma$-algebra.
\end{lemma}

\begin{lemma}[Directing measure is probability measure]
  \label{lem:directingMeasure_isProb}
  \lean{Exchangeability.DeFinetti.ViaMartingale.directingMeasure_isProb}
  \leanok
  \uses{def:tailSigma}
  The directing measure $\nu(\omega)(B) = \mathbb{E}[\mathbf{1}_{X_0 \in B} \mid \mathcal{T}](\omega)$
  is a probability measure for a.e.\ $\omega$.
\end{lemma}

\begin{lemma}[Directing measure measurability]
  \label{lem:directingMeasure_measurable}
  \lean{Exchangeability.DeFinetti.ViaMartingale.directingMeasure_measurable_eval}
  \leanok
  \uses{def:tailSigma}
  The directing measure $\omega \mapsto \nu(\omega)(B)$ is measurable for each Borel $B$.
\end{lemma}

\begin{theorem}[Contractable implies Conditionally i.i.d.\ (via Martingale)]
  \label{thm:conditionallyIID_of_contractable_martingale}
  \lean{Exchangeability.DeFinetti.conditionallyIID_of_contractable}
  \leanok
  \uses{def:contractable, def:conditionallyIID, lem:extreme_members_equal, lem:block_coord_condIndep, lem:directingMeasure_isProb, lem:directingMeasure_measurable, thm:condIndep_extension}
  If $(X_n)$ is contractable, then it is conditionally i.i.d.
  The directing kernel $\nu(\omega)(B) = \mathbb{E}[\mathbf{1}_{X_0 \in B} \mid \mathcal{T}](\omega)$
  is constructed from the tail $\sigma$-algebra.
\end{theorem}


\section{\texorpdfstring{Via L$^2$}{Via L²} (Elementary proof)}

The L$^2$ approach uses elementary contractability bounds on block averages.
This is Kallenberg's ``second proof'' and has the lightest dependencies.

\textbf{Note:} This proof applies to \emph{real-valued} sequences ($X : \mathbb{N} \to \Omega \to \mathbb{R}$)
with $L^2$ integrability (i.e., $\mathbb{E}[X_i^2] < \infty$ for all $i$).

\subsection{Block Averages and Covariance Structure}

\begin{definition}[Block average (L$^2$ version)]
  \label{def:blockAvg_L2}
  \lean{Exchangeability.DeFinetti.ViaL2.blockAvg}
  \leanok
  The block average $A_n = \frac{1}{n}\sum_{i=0}^{n-1} f(X_i)$ for bounded measurable $f$.
\end{definition}

\begin{lemma}[Contractable covariance structure]
  \label{lem:covariance_structure}
  \lean{Exchangeability.DeFinetti.ViaL2.contractable_covariance_structure}
  \leanok
  \uses{def:contractable, def:blockAvg_L2}
  For contractable sequences, the covariance $\mathrm{Cov}(f(X_i), f(X_j))$ is constant
  for $i \ne j$.
\end{lemma}

\begin{lemma}[L$^2$ bound on window differences]
  \label{lem:l2_bound_windows}
  \lean{Exchangeability.DeFinetti.ViaL2.l2_bound_two_windows}
  \leanok
  \uses{lem:covariance_structure}
  The L$^2$ norm of the difference between averages over disjoint windows is bounded.
\end{lemma}

\begin{lemma}[L$^2$ contractability bound]
  \label{lem:l2_bound}
  \lean{Exchangeability.DeFinetti.L2Approach.l2_contractability_bound}
  \leanok
  \uses{lem:l2_bound_windows}
  For contractable sequences, certain L$^2$ norms of block averages are bounded.
\end{lemma}

\subsection{Cesaro Convergence}

\begin{lemma}[Kallenberg L$^2$ bound]
  \label{lem:kallenberg_L2_bound}
  \lean{Exchangeability.DeFinetti.ViaL2.kallenberg_L2_bound}
  \leanok
  \uses{lem:l2_bound}
  The key L$^2$ bound that drives the Cesaro convergence.
\end{lemma}

\begin{lemma}[Cesaro to conditional expectation (L$^2$)]
  \label{lem:cesaro_L2}
  \lean{Exchangeability.DeFinetti.ViaL2.cesaro_to_condexp_L2}
  \leanok
  \uses{lem:kallenberg_L2_bound}
  Block averages converge in L$^2$ to the conditional expectation given the tail.
\end{lemma}

\begin{lemma}[Cesaro to conditional expectation (L$^1$)]
  \label{lem:cesaro_L1}
  \lean{Exchangeability.DeFinetti.ViaL2.cesaro_to_condexp_L1}
  \leanok
  \uses{lem:cesaro_L2}
  Block averages converge in L$^1$ to the conditional expectation given the tail.
\end{lemma}

\subsection{Directing Measure Construction}

\begin{lemma}[CDF from alpha bounds]
  \label{lem:cdf_bounds}
  \lean{Exchangeability.DeFinetti.ViaL2.cdf_from_alpha_bounds}
  \leanok
  The limiting function $\alpha(t, \omega)$ is a.e.\ in $[0,1]$.
\end{lemma}

\begin{lemma}[CDF monotonicity]
  \label{lem:cdf_mono}
  \lean{Exchangeability.DeFinetti.ViaL2.cdf_from_alpha_mono}
  \leanok
  The limiting function $\alpha(t, \omega)$ is monotone in $t$ for a.e.\ $\omega$.
\end{lemma}

\begin{lemma}[Directing measure is probability (L$^2$ version)]
  \label{lem:directing_measure_isProb_L2}
  \lean{Exchangeability.DeFinetti.ViaL2.directing_measure_isProbabilityMeasure}
  \leanok
  \uses{lem:cdf_bounds, lem:cdf_mono}
  The Stieltjes measure constructed from the limiting CDF is a probability measure a.e.
\end{lemma}

\begin{theorem}[Weighted sums converge in L$^1$]
  \label{thm:weighted_sums_L1}
  \lean{Exchangeability.DeFinetti.ViaL2.weighted_sums_converge_L1}
  \leanok
  \uses{lem:cesaro_L1}
  The weighted sums of indicators converge in L$^1$.
\end{theorem}

\begin{theorem}[Contractable implies Conditionally i.i.d.\ (via L$^2$)]
  \label{thm:conditionallyIID_of_contractable_L2}
  \lean{Exchangeability.DeFinetti.conditionallyIID_of_contractable_viaL2}
  \leanok
  \uses{def:contractable, def:conditionallyIID, lem:l2_bound, thm:weighted_sums_L1, lem:directing_measure_isProb_L2, thm:condIndep_extension}
  If $(X_n)$ is contractable, then it is conditionally i.i.d.
\end{theorem}


\section{Via Koopman (Mean Ergodic Theorem)}

The Koopman approach uses the Mean Ergodic Theorem via the shift operator on L$^2$.
This is Kallenberg's ``first proof'' and uses disjoint-block averaging.

\subsection{Block Averages and Ergodic Theory}

\begin{definition}[Block average]
  \label{def:blockAvg}
  \lean{Exchangeability.DeFinetti.ViaKoopman.blockAvg}
  \leanok
  The \emph{block average} $A_{m,n,k}(f)(\omega) = \frac{1}{n} \sum_{j=0}^{n-1} f(\omega_{k \cdot n + j})$
  averages $f$ over the $k$-th block of size $n$ (indices $[kn, kn+n)$).
  For $n = 0$, the block average is defined as $0$.
\end{definition}

\begin{lemma}[Koopman-condexp commutation]
  \label{lem:koopman_condexp_comm}
  \lean{Exchangeability.DeFinetti.ViaKoopman.condexpL2_koopman_comm}
  \leanok
  \uses{def:shiftInvariantSigma}
  The conditional expectation operator onto the shift-invariant subspace
  commutes with the Koopman operator.
\end{lemma}

\begin{theorem}[Birkhoff averages converge to condexp]
  \label{thm:birkhoff_condexp}
  \lean{Exchangeability.DeFinetti.ViaKoopman.birkhoffAverage_tendsto_condexp}
  \leanok
  \uses{lem:koopman_condexp_comm, def:shiftInvariantSigma}
  Birkhoff averages converge in L$^2$ to the conditional expectation
  given the shift-invariant $\sigma$-algebra.
\end{theorem}

\begin{lemma}[Block averages converge in $L^1$]
  \label{lem:blockAvg_tendsto}
  \lean{Exchangeability.DeFinetti.ViaKoopman.blockAvg_tendsto_condExp}
  \leanok
  \uses{def:blockAvg, thm:birkhoff_condexp}
  For a shift-invariant measure, block averages converge \emph{in $L^1$} to the conditional
  expectation given the shift-invariant $\sigma$-algebra:
  $\int |A_{m,n,k}(f) - \mathbb{E}[f \circ \pi_0 \mid \mathcal{I}]| \, d\mu \to 0$
  as $n \to \infty$.
\end{lemma}

\subsection{Contractability and Factorization}

\begin{lemma}[Conditional expectation lag constancy from exchangeability]
  \label{lem:condexp_lag_const}
  \lean{Exchangeability.DeFinetti.ViaKoopman.condexp_lag_constant_from_exchangeability}
  \leanok
  \uses{def:exchangeable, def:shiftInvariantSigma}
  For exchangeable sequences, the conditional expectation of a product
  does not depend on the lag between coordinates.
\end{lemma}

\begin{lemma}[Integral product equals block average product]
  \label{lem:integral_prod_blockAvg}
  \lean{Exchangeability.DeFinetti.ViaKoopman.integral_prod_eq_integral_blockAvg}
  \leanok
  \uses{def:blockAvg, def:contractable}
  For contractable sequences, integrals of products factor through block averages.
\end{lemma}

\begin{lemma}[Product block average L$^1$ convergence]
  \label{lem:product_blockAvg_L1}
  \lean{Exchangeability.DeFinetti.ViaKoopman.product_blockAvg_L1_convergence}
  \leanok
  \uses{lem:blockAvg_tendsto, lem:integral_prod_blockAvg}
  Products of block averages converge in L$^1$.
\end{lemma}

\begin{theorem}[Conditional expectation product factorization]
  \label{thm:condexp_product_factorization}
  \lean{Exchangeability.DeFinetti.ViaKoopman.condexp_product_factorization_contractable}
  \leanok
  \uses{lem:product_blockAvg_L1, lem:condexp_lag_const}
  For contractable sequences, the conditional expectation of a product
  of indicators factors as a product of conditional expectations.
\end{theorem}

\begin{lemma}[Bridge from contractability]
  \label{lem:indicator_product_bridge_contractable}
  \lean{Exchangeability.DeFinetti.indicator_product_bridge_contractable}
  \leanok
  \uses{def:contractable}
  For contractable sequences, indicator products satisfy the bridge condition.
\end{lemma}

\begin{theorem}[Contractable implies Conditionally i.i.d.\ (via Koopman)]
  \label{thm:conditionallyIID_of_contractable_Koopman}
  \lean{Exchangeability.DeFinetti.conditionallyIID_of_contractable_viaKoopman}
  \leanok
  \uses{def:contractable, def:conditionallyIID, lem:blockAvg_tendsto, thm:condexp_product_factorization, lem:indicator_product_bridge_contractable, thm:condIndep_extension}
  If $(X_n)$ is contractable, then it is conditionally i.i.d.
  This proof uses the Mean Ergodic Theorem via the Koopman operator on L$^2$.
\end{theorem}


\chapter{Common Ending}

All three proofs converge to the same final step: extending from indicators to general sets
via a monotone class argument.

\begin{lemma}[$\pi$-system uniqueness]
  \label{lem:pi_system}
  \lean{Exchangeability.measure_eq_of_fin_marginals_eq}
  \leanok
  Measures on product spaces are determined by their finite-dimensional marginals.
\end{lemma}

\begin{theorem}[Monotone class theorem]
  \label{thm:monotone_class}
  \lean{Exchangeability.DeFinetti.CommonEnding.monotone_class_theorem}
  \leanok
  \uses{lem:pi_system}
  The monotone class theorem allows extending from indicators to measurable functions.
\end{theorem}

\begin{theorem}[Conditional independence extension]
  \label{thm:condIndep_extension}
  \lean{Exchangeability.DeFinetti.CommonEnding.conditional_iid_from_directing_measure}
  \leanok
  \uses{thm:monotone_class, lem:pi_system}
  Conditional independence on indicators extends to the full product $\sigma$-algebra.
\end{theorem}


\chapter{Main Theorem}

\begin{theorem}[de Finetti--Ryll-Nardzewski equivalence]
  \label{thm:deFinetti}
  \lean{Exchangeability.DeFinetti.deFinetti_RyllNardzewski_equivalence}
  \leanok
  \uses{thm:contractable_of_exchangeable, thm:exchangeable_of_conditionallyIID, thm:conditionallyIID_of_contractable_martingale, thm:conditionallyIID_of_contractable_L2, thm:conditionallyIID_of_contractable_Koopman}
  For an infinite sequence $(X_n)_{n \in \mathbb{N}}$ of random variables taking values
  in a \emph{standard Borel space} $\alpha$ (with $\alpha$ nonempty), the following are equivalent:
  \begin{enumerate}
    \item $(X_n)$ is contractable
    \item $(X_n)$ is exchangeable
    \item $(X_n)$ is conditionally i.i.d.\ (i.e., there exists a directing kernel $\nu$)
  \end{enumerate}
  \textbf{Remark:} The martingale proof constructs $\nu$ from the tail $\sigma$-algebra $\mathcal{T}$
  via $\nu(\omega)(B) = \mathbb{E}[\mathbf{1}_{X_0 \in B} \mid \mathcal{T}](\omega)$.
\end{theorem}
