\section{Route 3: Mean Ergodic Theorem}
\label{sec:koopman}

The Koopman proof is Kallenberg's ``first proof.'' It connects de Finetti's theorem to \textbf{ergodic theory}, revealing that exchangeability is fundamentally about dynamical invariance under the shift operator.

\subsection{Key Ideas}

The proof interprets the sequence as a dynamical system:
\begin{enumerate}
\item The \textbf{shift operator} $T : (\N \to \alpha) \to (\N \to \alpha)$ with $(Tx)_n = x_{n+1}$ is measure-preserving when $X$ is contractable.
\item The \textbf{Koopman operator} $U_T : L^2(\mu) \to L^2(\mu)$ with $(U_T f)(\omega) = f(T\omega)$ is an isometry.
\item The \textbf{Mean Ergodic Theorem} asserts that Ces\`{a}ro averages converge in $L^2$ to the projection onto $U_T$-invariant functions.
\end{enumerate}

\subsection{The Koopman Operator}

\begin{definition}
For a measure-preserving transformation $T : \Omega \to \Omega$, the Koopman operator is:
\[
U_T : L^2(\mu) \to L^2(\mu), \quad (U_T f)(\omega) = f(T\omega)
\]
\end{definition}

\begin{lstlisting}
def koopman {μ : Measure Ω} [IsProbabilityMeasure μ]
    (T : Ω → Ω) (hT : MeasurePreserving T μ μ) :
    Lp ℝ 2 μ →L[ℝ] Lp ℝ 2 μ :=
  (MeasureTheory.Lp.compMeasurePreservingₗᵢ ℝ T hT).toContinuousLinearMap
\end{lstlisting}

Since $T$ is measure-preserving, $U_T$ is an isometry: $\|U_T f\|_2 = \|f\|_2$.

\subsection{Mean Ergodic Theorem}

\begin{theorem}[von Neumann, 1932]
Let $U : L^2(\mu) \to L^2(\mu)$ be an isometry. Then for any $f \in L^2(\mu)$:
\[
\frac{1}{n} \sum_{i=0}^{n-1} U^i f \to Pf \quad \text{in } L^2
\]
where $P$ is the orthogonal projection onto the closed subspace $\{g : Ug = g\}$ of $U$-invariant functions.
\end{theorem}

\begin{lstlisting}
theorem birkhoffAverage_tendsto_metProjection
    {μ : Measure Ω} [IsProbabilityMeasure μ] (T : Ω → Ω)
    (hT : MeasurePreserving T μ μ) (f : Lp ℝ 2 μ) :
    Tendsto (fun n => birkhoffAverage ℝ (koopman T hT) _root_.id n f)
      atTop (nhds (metProjection T hT f))
\end{lstlisting}

\subsection{Invariant Functions and the Tail}

A function $f$ is $U_T$-invariant if $f(T\omega) = f(\omega)$ a.e.\ This is equivalent to $f$ being measurable with respect to the \textbf{shift-invariant $\sigma$-algebra}:
\[
\mathcal{I} = \{A : T^{-1}(A) = A \text{ a.e.}\}
\]

For the shift on path space, the shift-invariant $\sigma$-algebra is closely related to the tail $\sigma$-algebra $\cF_\infty = \bigcap_n \sigma(X_n, X_{n+1}, \ldots)$. (The formal equivalence is a standard result used but not re-proved in this formalization.)

\subsection{Proof Skeleton}

\paragraph{Step 1: Shift is measure-preserving.}
Contractability implies $T_*\mu = \mu$ for the shift operator.

\paragraph{Step 2: Block averaging via contractability.}
For $m$ disjoint blocks, contractability gives:
\[
\int \prod_i f_i(X_i) \, d\mu = \int \prod_i (\text{block average of } f_i) \, d\mu
\]

\paragraph{Step 3: $L^2$ convergence via Mean Ergodic Theorem.}
Block averages converge in $L^2$ to conditional expectations given the shift-invariant $\sigma$-algebra.

\paragraph{Step 4: Product factorization.}
Taking the limit:
\[
\E[\prod_i f_i(X_i) | \mathcal{I}] = \prod_i \E[f_i(X_0) | \mathcal{I}] \quad \text{a.e.}
\]

\paragraph{Step 5: Construct directing measure.}
Define $\nu(\omega) = \Law(X_0 | \mathcal{I})(\omega)$.

\paragraph{Step 6: Extend via $\pi$-system.}
Common ending (Section~\ref{sec:common-ending}).

\subsection{Key Lemmas}

\begin{center}
\begin{tabular}{ll}
\textbf{Lemma} & \textbf{File:Line} \\
\hline
\texttt{pathSpace\_shift\_preserving\_of\_contractable} & \texttt{ViaKoopman.lean:385} \\
\texttt{koopman\_isometry} & \texttt{KoopmanMeanErgodic.lean:130} \\
\texttt{birkhoffAverage\_tendsto\_metProjection} & \texttt{KoopmanMeanErgodic.lean:245} \\
\texttt{condexp\_product\_factorization\_contractable} & \texttt{ContractableFactorization.lean:477} \\
\texttt{conditionallyIID\_of\_contractable\_viaKoopman} & \texttt{TheoremViaKoopman.lean:224} \\
\end{tabular}
\end{center}

\subsection{Mathematical Significance}

This proof reveals de Finetti's theorem as part of \textbf{ergodic theory}:
\begin{itemize}
\item Exchangeability means the shift dynamics is measure-preserving.
\item The directing measure arises from the ergodic decomposition.
\item The Mean Ergodic Theorem provides the convergence.
\item Conditionally on the invariant $\sigma$-algebra, coordinates are i.i.d.
\end{itemize}

This connection places de Finetti's theorem in a broader context alongside the Birkhoff Ergodic Theorem and ergodic decomposition theory.
