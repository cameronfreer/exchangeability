\section{Architecture, Verification, and Usability}
\label{sec:short-sections}

This section covers practical aspects of the formalization: code architecture, verification methodology, and the user-facing API.

\subsection{Architecture}

\paragraph{Module structure.}
The codebase is organized into logical modules:

\begin{verbatim}
Exchangeability/
  Contractability.lean       -- Exchangeable, Contractable, easy implications
  ConditionallyIID.lean      -- ConditionallyIID definition and lemmas
  Core.lean                  -- Pi-system machinery, cylinder sets
  Tail/                      -- Tail sigma-algebra infrastructure
  Probability/               -- Conditional expectation, kernels, independence
  Ergodic/                   -- Koopman operators, Mean Ergodic Theorem
  PathSpace/                 -- Shift operator on sequences
  DeFinetti/
    Theorem.lean             -- Public API (re-exports martingale proof)
    TheoremViaMartingale.lean
    TheoremViaL2.lean
    TheoremViaKoopman.lean
    CommonEnding.lean        -- Shared pi-system extension
    ViaMartingale/           -- Martingale proof internals
    ViaL2/                   -- L2 proof internals
    ViaKoopman/              -- Koopman proof internals
\end{verbatim}

\paragraph{Separation of concerns.}
Each proof route is self-contained in its subdirectory. Shared infrastructure (definitions, easy directions, common ending) is factored out. This design allows:
\begin{itemize}
\item Independent development and testing of each route.
\item Clear identification of what each route contributes.
\item Easy addition of future proof routes.
\end{itemize}

\paragraph{Public vs.\ internal API.}
The public API is exposed through \texttt{DeFinetti/Theorem.lean}:
\begin{lstlisting}
-- Main theorem
theorem deFinetti : Exchangeable mu X -> ConditionallyIID mu X

-- Equivalence
theorem deFinetti_equivalence : Exchangeable mu X <-> ConditionallyIID mu X

-- Full Kallenberg 1.1
theorem deFinetti_RyllNardzewski_equivalence :
    Contractable mu X <-> Exchangeable mu X /\ ConditionallyIID mu X
\end{lstlisting}

Internal lemmas are not exported, keeping the public interface clean.

\subsection{Verification}

\paragraph{Build verification.}
The entire library builds successfully with \texttt{lake build}:
\begin{verbatim}
$ lake build
Build completed successfully (3296 jobs).
\end{verbatim}

\paragraph{Axiom hygiene.}
All theorems depend only on standard mathlib axioms:
\begin{lstlisting}
#print axioms Exchangeability.DeFinetti.deFinetti
-- 'Exchangeability.DeFinetti.deFinetti' depends on axioms:
--   [propext, Classical.choice, Quot.sound]
\end{lstlisting}

These are:
\begin{itemize}
\item \texttt{propext}: Propositional extensionality (unavoidable in mathlib)
\item \texttt{Classical.choice}: Axiom of choice (required for measure theory)
\item \texttt{Quot.sound}: Quotient soundness (standard for quotient types)
\end{itemize}

No custom axioms are introduced. No \texttt{sorry} appears in any theorem.

\paragraph{Introspection file.}
The file \texttt{paper\_materials/PaperIntrospection.lean} provides machine-checkable verification:
\begin{lstlisting}
#check @Exchangeability.DeFinetti.deFinetti
#check @Exchangeability.DeFinetti.deFinetti_equivalence
#print axioms Exchangeability.DeFinetti.deFinetti
\end{lstlisting}

\subsection{Usability}

\paragraph{Quick start.}
To use the main theorem:
\begin{lstlisting}
import Exchangeability.DeFinetti.Theorem

variable {Omega : Type*} [MeasurableSpace Omega] [StandardBorelSpace Omega]
variable {alpha : Type*} [MeasurableSpace alpha] [StandardBorelSpace alpha] [Nonempty alpha]
variable {mu : Measure Omega} [IsProbabilityMeasure mu]
variable (X : N -> Omega -> alpha) (hX_meas : forall i, Measurable (X i))

example (hExch : Exchangeable mu X) : ConditionallyIID mu X :=
  Exchangeability.DeFinetti.deFinetti X hX_meas hExch
\end{lstlisting}

\paragraph{Working with \texttt{ConditionallyIID}.}
The \texttt{ConditionallyIID} predicate is a \texttt{def} (existential), not a \texttt{structure}. To extract the directing measure:
\begin{lstlisting}
variable (hCIID : ConditionallyIID mu X)

-- Get the directing measure
def nu : Omega -> Measure alpha := hCIID.choose

-- It's a probability measure
example (omega : Omega) : IsProbabilityMeasure (nu omega) :=
  hCIID.choose_spec.1 omega
\end{lstlisting}

\paragraph{Alternative proof routes.}
For specialized applications:
\begin{lstlisting}
-- L2 proof (for R-valued, L2 integrable)
import Exchangeability.DeFinetti.TheoremViaL2
#check conditionallyIID_of_contractable_viaL2

-- Koopman proof (for R-valued, L2 integrable)
import Exchangeability.DeFinetti.TheoremViaKoopman
#check conditionallyIID_of_contractable_viaKoopman
\end{lstlisting}

\subsection{Statistics}

\begin{center}
\begin{tabular}{lr}
\textbf{Metric} & \textbf{Value} \\
\hline
Total Lean files & 67 \\
Total lines of Lean code & $\sim$16,500 \\
Build jobs & 3,296 \\
Axioms used & 3 (standard) \\
\end{tabular}
\end{center}
