\section{Comparison of the Three Proofs}
\label{sec:comparison}

Having formalized three independent proofs of the same theorem, we can compare them systematically across several dimensions.

\subsection{Summary Table}

\begin{center}
\begin{tabular}{lccc}
\textbf{Aspect} & \textbf{Martingale} & \textbf{$L^2$} & \textbf{Koopman} \\
\hline
Reference & Kallenberg ``Third'' & Kallenberg ``Second'' & Kallenberg ``First'' \\
Key technique & Reverse martingale & Correlation bounds & Mean Ergodic Thm \\
Dependencies & Medium & Lightest & Heaviest \\
State space & Standard Borel & $\R$ ($L^2$) & $\R$ ($L^2$) \\
Files & 14 & 13 & 19 \\
Lines & $\sim$4200 & $\sim$12500 & $\sim$7500 \\
Default & Yes & No & No \\
\end{tabular}
\end{center}

\subsection{Conceptual Comparison}

\paragraph{Martingale route.}
The martingale proof is \textbf{probabilistically natural}: it uses conditional expectations, reverse martingales, and the tail $\sigma$-algebra---concepts central to probability theory. The directing measure is constructed as the conditional distribution of $X_0$ given the tail, which is the ``correct'' probabilistic answer to ``what does $X$ look like in the limit?''

\paragraph{$L^2$ route.}
The $L^2$ proof is \textbf{elementary}: it avoids both ergodic theory and reverse martingale convergence. The key insight is that contractability forces correlations to decay, which is captured by explicit $L^2$ bounds. This proof is closest to a ``bare hands'' approach.

\paragraph{Koopman route.}
The Koopman proof is \textbf{conceptually deep}: it reveals de Finetti's theorem as a manifestation of ergodic theory. The shift operator provides a dynamical perspective, and the Mean Ergodic Theorem connects to von Neumann's foundational work on ergodic averages. This proof places exchangeability in a broader mathematical context.

\subsection{Dependency Comparison}

\begin{center}
\begin{tabular}{lccc}
\textbf{mathlib Component} & \textbf{Martingale} & \textbf{$L^2$} & \textbf{Koopman} \\
\hline
Conditional expectation & Heavy & Light & Medium \\
Martingale theory & Medium & None & None \\
$L^2$ spaces & Light & Heavy & Heavy \\
Hilbert space projections & None & None & Heavy \\
Ergodic theory & None & None & Heavy \\
Probability kernels & Heavy & Light & Light \\
\end{tabular}
\end{center}

\subsection{Generalizability}

\begin{itemize}
\item \textbf{Martingale}: Works for arbitrary standard Borel state spaces. No integrability assumptions needed.

\item \textbf{$L^2$}: Requires $\R$-valued (or Hilbert space-valued) random variables with $L^2$ integrability. Can handle general state spaces via Borel isomorphism to $\R$, but this adds complexity.

\item \textbf{Koopman}: Similar to $L^2$---requires $L^2$ integrability. The ergodic-theoretic framework is more naturally suited to $\R$-valued functions.
\end{itemize}

\subsection{What We Learned}

Formalizing three proofs of the same theorem taught us several lessons:

\paragraph{Shared infrastructure matters.}
The easy directions and common ending are shared across all three routes, comprising roughly 30\% of the codebase. This sharing reduces duplication and ensures consistency.

\paragraph{Different proofs stress different parts of mathlib.}
The martingale proof exercises probability kernels and conditional expectation. The $L^2$ proof exercises $L^p$ spaces and Cauchy sequences. The Koopman proof exercises Hilbert space projections and ergodic theory. This provides good coverage of mathlib's probability infrastructure.

\paragraph{Proof length varies with mathematical style.}
The $L^2$ proof has more lines despite being ``elementary'' because explicit bounds require more bookkeeping. The martingale proof is shorter because it leverages more powerful abstractions.

\paragraph{Choosing a default is a design decision.}
We chose the martingale proof as the default because it is the most general (arbitrary state spaces) and has reasonable dependency weight. A different library might prefer the $L^2$ proof for its lighter dependencies.

\subsection{Recommendations}

\begin{itemize}
\item \textbf{For general use}: Import \texttt{DeFinetti.Theorem}, which uses the martingale proof.
\item \textbf{For understanding dependencies}: Study the $L^2$ proof.
\item \textbf{For connections to dynamics}: Study the Koopman proof.
\item \textbf{For mathlib contribution}: The martingale proof is likely the cleanest candidate.
\end{itemize}
