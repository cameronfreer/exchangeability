\section{Route 2: Elementary $L^2$ Bounds}
\label{sec:l2}

The $L^2$ proof is Kallenberg's ``second proof.'' It has the \textbf{lightest dependencies}---no ergodic theory, minimal martingale theory---using only $L^2$ spaces and elementary correlation bounds.

\subsection{Key Ideas}

The central observation is Kallenberg's Lemma~1.2:

\begin{lemma}[Correlation bound]
For a contractable sequence $(X_i)$ of bounded random variables:
\[
|\E[X_i \cdot X_j] - \E[X_i] \cdot \E[X_j]| \le \frac{C}{\min(i+1, j+1)}
\]
for some constant $C$ depending on the bounds.
\end{lemma}

This bound implies that the correlation between distant terms decays, which forces the block averages to converge.

\subsection{Block Averages}

Define the block average:
\[
\alpha_n(\omega) = \frac{1}{n} \sum_{i=0}^{n-1} X_i(\omega)
\]

\begin{lstlisting}
def blockAvg (X : N -> Omega -> R) (n : N) (omega : Omega) : R :=
  (1 / n) * sum i in Finset.range n, X i omega
\end{lstlisting}

\subsection{$L^2$ Convergence}

The correlation bound implies that $(\alpha_n)$ is Cauchy in $L^2$:
\[
\E[(\alpha_n - \alpha_m)^2] \le C \cdot \left(\frac{\log n}{n} + \frac{\log m}{m}\right)
\]

Therefore, there exists $\alpha_\infty \in L^2(\mu)$ with $\alpha_n \to \alpha_\infty$ in $L^2$.

\subsection{Proof Skeleton}

\paragraph{Step 1: Clip to $[0,1]$.}
For unbounded random variables, first work with $\mathrm{clip}_{01}(X) = \max(0, \min(1, X))$. The general case follows by approximation.

\paragraph{Step 2: Establish the correlation bound.}
From contractability: $(X_0, X_i) \deq (X_j, X_i)$ for $j \le i$. This bounds cross-correlations.

\paragraph{Step 3: Prove $L^2$ convergence of block averages.}
The correlation bound implies $\|\alpha_n - \alpha_m\|_2 \to 0$ as $n, m \to \infty$.

\paragraph{Step 4: Product factorization.}
For bounded measurable $f, g : \R \to \R$:
\[
\E[f(X_i) \cdot g(X_j)] = \E[\E[f(X_0) | \alpha_\infty] \cdot \E[g(X_0) | \alpha_\infty]]
\]
This is the key factorization showing conditional independence.

\paragraph{Step 5: Construct directing measure.}
From the factorization, construct $\nu : \Omega \to \mathrm{Measure}(\R)$ with:
\[
\int f \, d\nu(\omega) = \E[f(X_0) | \alpha_\infty](\omega)
\]

\paragraph{Step 6: Extend via $\pi$-system.}
Common ending (Section~\ref{sec:common-ending}).

\subsection{Key Lemmas}

\begin{center}
\begin{tabular}{ll}
\textbf{Lemma} & \textbf{File:Line} \\
\hline
\texttt{blockAvg} & \texttt{BlockAvgDef.lean:45} \\
\texttt{l2\_contractability\_bound} & \texttt{L2Helpers.lean:852} \\
\texttt{reverse\_martingale\_subsequence\_convergence} & \texttt{MainConvergence.lean:796} \\
\texttt{conditionallyIID\_of\_contractable\_viaL2} & \texttt{TheoremViaL2.lean:135} \\
\end{tabular}
\end{center}

\subsection{Why ``Elementary''?}

This proof is called ``elementary'' because it:
\begin{itemize}
\item Uses \textbf{no ergodic theory} (no Koopman operators or Mean Ergodic Theorem).
\item Uses \textbf{no reverse martingale convergence} in the technical sense.
\item Relies only on \textbf{$L^2$ Hilbert space structure}.
\item Has \textbf{explicit, computable bounds}.
\end{itemize}

\subsection{Limitations}

The $L^2$ proof works directly only for \textbf{$\R$-valued} random variables (or more generally, Hilbert space-valued). For general standard Borel spaces, one must either:
\begin{itemize}
\item Embed into $\R$ via a Borel isomorphism, or
\item Use the martingale proof instead.
\end{itemize}

In the formalization, the main theorem requires $L^2$ integrability:
\begin{lstlisting}
theorem conditionallyIID_of_contractable_viaL2
    [StandardBorelSpace Omega]
    (mu : Measure Omega) [IsProbabilityMeasure mu]
    (X : N -> Omega -> R) (hX_meas : forall i, Measurable (X i))
    (hContract : Contractable mu X)
    (hX_L2 : forall i, MemLp (X i) 2 mu) :
    ConditionallyIID mu X
\end{lstlisting}
