\section{Definitions}
\label{sec:definitions}

We work in a standard measure-theoretic setting. Let $\Omega$ be a probability space with measure $\mu$, and let $\alpha$ be a standard Borel space (the state space). A \emph{stochastic process} is a sequence $X : \N \to \Omega \to \alpha$ of measurable functions.

\subsection{Exchangeable}

\begin{definition}[Exchangeable]
A sequence $X : \N \to \Omega \to \alpha$ is \emph{exchangeable} (under $\mu$) if for every $n \in \N$ and every permutation $\sigma$ of $\{0, 1, \ldots, n-1\}$:
\[
(X_{\sigma(0)}, X_{\sigma(1)}, \ldots, X_{\sigma(n-1)}) \deq (X_0, X_1, \ldots, X_{n-1}).
\]
\end{definition}

In Lean, this is expressed as equality of pushforward measures:
\begin{lstlisting}
def Exchangeable (μ : Measure Ω) (X : ℕ → Ω → α) : Prop :=
  ∀ n (σ : Equiv.Perm (Fin n)),
    Measure.map (fun ω => fun i : Fin n => X (σ i) ω) μ =
      Measure.map (fun ω => fun i : Fin n => X i ω) μ
\end{lstlisting}

\subsection{Contractable}

\begin{definition}[Contractable]
A sequence $X$ is \emph{contractable} if every strictly increasing subsequence has the same distribution as the initial segment of the same length. Formally, for every $m \in \N$ and every strictly increasing $k : \{0, \ldots, m-1\} \to \N$:
\[
(X_{k(0)}, X_{k(1)}, \ldots, X_{k(m-1)}) \deq (X_0, X_1, \ldots, X_{m-1}).
\]
\end{definition}

\begin{lstlisting}
def Contractable (μ : Measure Ω) (X : ℕ → Ω → α) : Prop :=
  ∀ (m : ℕ) (k : Fin m → ℕ), StrictMono k →
    Measure.map (fun ω i => X (k i) ω) μ =
      Measure.map (fun ω i => X i.val ω) μ
\end{lstlisting}

\begin{remark}
Contractability is ostensibly weaker than exchangeability: it only requires invariance under ``thinning'' (selecting a subsequence in order), not under arbitrary reordering. The main theorem shows these conditions are equivalent for infinite sequences on standard Borel spaces.
\end{remark}

\subsection{Conditionally IID}

\begin{definition}[Conditionally IID]
A sequence $X$ is \emph{conditionally i.i.d.}\ if there exists a \emph{directing measure} $\nu : \Omega \to \mathrm{Measure}(\alpha)$ such that:
\begin{enumerate}
\item For each $\omega$, $\nu(\omega)$ is a probability measure on $\alpha$.
\item The map $\omega \mapsto \nu(\omega)(B)$ is measurable for each measurable $B \subseteq \alpha$.
\item For every strictly increasing $k : \{0, \ldots, m-1\} \to \N$:
\[
\Law(X_{k(0)}, \ldots, X_{k(m-1)}) = \int \nu(\omega)^{\otimes m} \, d\mu(\omega).
\]
\end{enumerate}
\end{definition}

\begin{lstlisting}
def ConditionallyIID (μ : Measure Ω) (X : ℕ → Ω → α) : Prop :=
  ∃ ν : Ω → Measure α,
    (∀ ω, IsProbabilityMeasure (ν ω)) ∧
    (∀ B, MeasurableSet B → Measurable (fun ω => ν ω B)) ∧
      ∀ (m : ℕ) (k : Fin m → ℕ), StrictMono k →
        Measure.map (fun ω => fun i : Fin m => X (k i) ω) μ
          = μ.bind (fun ω => Measure.pi fun _ : Fin m => ν ω)
\end{lstlisting}

The third condition says that the finite-dimensional distributions are mixtures of product measures.

\subsection{The Equivalence}

\begin{theorem}[de Finetti--Ryll-Nardzewski]
\label{thm:main}
Let $X : \N \to \Omega \to \alpha$ be a sequence of measurable random variables on a standard Borel probability space, taking values in a standard Borel space $\alpha$. Then:
\[
\text{Contractable} \iff \text{Exchangeable} \iff \text{Conditionally i.i.d.}
\]
\end{theorem}

The equivalence breaks into implications:
\begin{center}
\begin{tabular}{lll}
Direction & Difficulty & Method \\
\hline
Exchangeable $\Rightarrow$ Contractable & Easy & Permutation extension \\
Conditionally IID $\Rightarrow$ Exchangeable & Easy & Product measure invariance \\
Contractable $\Rightarrow$ Conditionally IID & Hard & Three proof routes \\
\end{tabular}
\end{center}

\paragraph{Lean API.}
The public API exposes two theorems. First, the equivalence \texttt{Exchangeable $\leftrightarrow$ ConditionallyIID}:
\begin{lstlisting}
theorem deFinetti_equivalence : Exchangeable μ X ↔ ConditionallyIID μ X
\end{lstlisting}
Second, the full three-way equivalence as stated by Kallenberg (Theorem 1.1):
\begin{lstlisting}
theorem deFinetti_RyllNardzewski_equivalence :
    Contractable μ X ↔ Exchangeable μ X ∧ ConditionallyIID μ X
\end{lstlisting}
Combined with \texttt{deFinetti\_equivalence}, this yields the three-way equivalence of Theorem~\ref{thm:main}.

\paragraph{Returning to the example.}
For the exchangeable coin-flip sequence from Section~\ref{sec:introduction}, the directing measure $\nu(\omega)$ is the Bernoulli distribution with parameter $\theta(\omega)$---the ``latent bias.'' The theorem says every exchangeable $\{0,1\}$-sequence arises as a mixture of such Bernoulli product measures.
