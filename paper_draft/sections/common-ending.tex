\section{Common Ending: $\pi$-System Extension}
\label{sec:common-ending}

All three proof routes share a common final step: extending the finite-dimensional product formula from cylinder sets to all Borel sets using \textbf{$\pi$-system uniqueness}.

\subsection{The $\pi$-System Theorem}

\begin{theorem}[Dynkin's $\pi$-$\lambda$ theorem]
Let $\mathcal{P}$ be a $\pi$-system (closed under finite intersections) that generates a $\sigma$-algebra $\cF$. If $\mu$ and $\nu$ are finite measures with $\mu(A) = \nu(A)$ for all $A \in \mathcal{P}$, then $\mu = \nu$ on $\cF$.
\end{theorem}

This theorem is crucial because checking equality on a $\pi$-system (often much smaller than the full $\sigma$-algebra) suffices to establish equality of measures.

\subsection{Cylinder Sets}

For the product space $\N \to \alpha$, cylinder sets are sets depending on finitely many coordinates:
\[
C_{n,S} = \{x \in \N \to \alpha : (x_0, \ldots, x_{n-1}) \in S\}
\]
for a measurable $S \subseteq \alpha^n$.

\begin{lstlisting}
def prefixProj (alpha : Type*) (n : N) (x : N -> alpha) : Fin n -> alpha :=
  fun i => x i

def prefixCylinder {n : N} (S : Set (Fin n -> alpha)) : Set (N -> alpha) :=
  (prefixProj alpha n)^(-1)' S
\end{lstlisting}

\begin{lemma}
Cylinder sets form a $\pi$-system that generates the product $\sigma$-algebra on $\N \to \alpha$.
\end{lemma}

\subsection{Application to de Finetti}

The three proof routes establish the product formula for cylinder sets:
\[
\mu(\{X \in C\}) = \int \nu(\omega)^{\otimes n}(C) \, d\mu(\omega)
\]
for cylinders $C = C_{n,S}$.

The $\pi$-system theorem then implies:
\[
\Law(X_0, X_1, \ldots, X_{n-1}) = \int \nu(\omega)^{\otimes n} \, d\mu(\omega)
\]
for \emph{all} Borel sets in $\alpha^n$, not just those of the form $B_0 \times \cdots \times B_{n-1}$.

\subsection{The Bridge to \texttt{ConditionallyIID}}

\begin{lstlisting}
theorem measure_eq_of_fin_marginals_eq
    {mu nu : Measure (N -> alpha)} [IsFiniteMeasure mu] [IsFiniteMeasure nu]
    (h : forall n S, MeasurableSet S -> mu (prefixCylinder S) = nu (prefixCylinder S)) :
    mu = nu
\end{lstlisting}

This lemma, combined with the finite-dimensional product formula for cylinders, yields the full \texttt{ConditionallyIID} condition:

\begin{lstlisting}
theorem conditionallyIID_of_finite_product_formula
    {nu : Omega -> Measure alpha}
    (h_nu_prob : forall omega, IsProbabilityMeasure (nu omega))
    (h_nu_meas : forall B, MeasurableSet B -> Measurable (fun omega => nu omega B))
    (h_product : forall m (k : Fin m -> N), StrictMono k ->
      forall (S : Set (Fin m -> alpha)), MeasurableSet S ->
        mu (cylinder_X S) = integral mu (fun omega => (Measure.pi (fun _ => nu omega)) S)) :
    ConditionallyIID mu X
\end{lstlisting}

\subsection{Relationship to Kolmogorov Extension}

The $\pi$-system approach is related to, but distinct from, the \textbf{Kolmogorov extension theorem}:
\begin{itemize}
\item \textbf{Kolmogorov extension} constructs a measure from consistent finite-dimensional marginals.
\item \textbf{$\pi$-system uniqueness} proves equality of two existing measures.
\end{itemize}

For de Finetti, we already have the measure $\mu$; we need to prove it equals the mixture of product measures. The $\pi$-system approach is more direct.

\subsection{Implementation Notes}

The common ending is implemented in \texttt{DeFinetti/CommonEnding.lean}. Each proof route calls into this shared infrastructure after establishing the cylinder-level product formula. This design ensures:
\begin{itemize}
\item No duplication of the $\pi$-system extension argument.
\item Clean separation between route-specific and shared code.
\item Easy comparison of what each route contributes.
\end{itemize}
