\section{Introduction}
\label{sec:introduction}

De Finetti's theorem is a foundational result in probability theory that characterizes exchangeable sequences. In its modern form, due to Ryll-Nardzewski~\citep{ryllnardzewski1957} and Hewitt--Savage~\citep{hewittsavage1955}, the theorem states that an infinite sequence of random variables is exchangeable if and only if it is conditionally independent and identically distributed given its tail $\sigma$-algebra.

\subsection{The Theorem}

Let $(X_0, X_1, X_2, \ldots)$ be a sequence of random variables taking values in a standard Borel space $\alpha$. The sequence is \emph{exchangeable} if its joint distribution is invariant under finite permutations of the indices:
\[
(X_{\sigma(0)}, X_{\sigma(1)}, \ldots, X_{\sigma(n-1)}) \deq (X_0, X_1, \ldots, X_{n-1})
\]
for every $n \in \N$ and every permutation $\sigma$ of $\{0, 1, \ldots, n-1\}$.

De Finetti's theorem asserts that exchangeability is equivalent to the existence of a \emph{directing measure}---a random probability measure $\nu$ on $\alpha$ such that, conditionally on $\nu$, the sequence $(X_i)$ is i.i.d.\ with distribution $\nu$. Formally:
\[
\Law(X_{k(0)}, \ldots, X_{k(m-1)}) = \int \nu(\omega)^{\otimes m} \, d\mu(\omega)
\]
for every strictly increasing sequence $k : \{0, \ldots, m-1\} \to \N$.

\subsection{Why Formalize De Finetti?}

De Finetti's theorem is significant for several reasons:

\paragraph{Foundational importance.} The theorem provides the mathematical foundation for Bayesian statistics, justifying the use of prior distributions as encoding ``unknown parameters'' that govern i.i.d.\ sampling.

\paragraph{Mathematical depth.} The proof requires substantial machinery: measure theory, conditional expectations, martingale convergence, and (depending on the proof route) ergodic theory or $L^2$ analysis.

\paragraph{Multiple proof techniques.} Kallenberg~\citep{kallenberg2005} presents three distinct proofs, each using different mathematical machinery. Formalizing all three illuminates the relationships between probabilistic, analytic, and dynamical approaches.

\paragraph{Verification of a classical result.} Despite its importance, de Finetti's theorem involves subtle measure-theoretic arguments that benefit from machine verification.

\subsection{Contributions}

This paper describes a Lean~4 formalization with the following contributions:

\begin{enumerate}
\item \textbf{Three independent proofs} of the hard direction (exchangeable $\Rightarrow$ conditionally i.i.d.), following Kallenberg's three approaches.

\item \textbf{A unified interface} through which all three proofs deliver the same theorem statement, enabling comparison and substitution.

\item \textbf{Clean separation} between the easy directions (which are shared) and the hard direction (which varies by proof route).

\item \textbf{Axiom hygiene}: all theorems depend only on \texttt{propext}, \texttt{Classical.choice}, and \texttt{Quot.sound}---the standard axioms used throughout mathlib.

\item \textbf{Lessons learned} about managing multiple proof strategies in a proof assistant, and about the interplay between different mathematical styles.
\end{enumerate}

\subsection{Paper Outline}

Section~\ref{sec:definitions} presents the formal definitions. Section~\ref{sec:roadmap} gives a roadmap of the proof structure. Sections~\ref{sec:martingale}, \ref{sec:l2}, and~\ref{sec:koopman} detail the three proof routes. Section~\ref{sec:common-ending} describes the shared $\pi$-system extension. Section~\ref{sec:comparison} compares the three approaches. Section~\ref{sec:short-sections} covers architecture, verification, and usability. Section~\ref{sec:related-work} discusses related work. Section~\ref{sec:conclusion} concludes with lessons learned.
