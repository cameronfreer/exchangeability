\section{Related Work}
\label{sec:related-work}

\subsection{Mathematical Background}

De Finetti's theorem has a rich history:
\begin{itemize}
\item \textbf{de Finetti}~\citep{definetti1931,definetti1937} proved the original result for $\{0,1\}$-valued sequences.
\item \textbf{Hewitt--Savage}~\citep{hewittsavage1955} extended the theorem to general state spaces using symmetric measures.
\item \textbf{Ryll-Nardzewski}~\citep{ryllnardzewski1957} gave an alternative extension using ergodic theory.
\item \textbf{Aldous}~\citep{aldous1985} provided a comprehensive survey including the martingale proof.
\item \textbf{Kallenberg}~\citep{kallenberg2005} unified the theory and presented the three proofs we formalize.
\end{itemize}

For general background on probability theory and measure theory, we refer to Kallenberg~\citep{kallenberg2002}.

\subsection{Formalizations of Probability Theory}

\paragraph{mathlib.}
The Lean mathematical library~\citep{mathlib} provides extensive infrastructure for measure theory and probability. Key components we use include:
\begin{itemize}
\item Measure spaces and integration (\texttt{MeasureTheory})
\item $L^p$ spaces (\texttt{MeasureTheory.Function.LpSpace})
\item Conditional expectation (\texttt{MeasureTheory.Function.ConditionalExpectation})
\item Probability kernels (\texttt{Probability.Kernel})
\item Martingale theory (\texttt{Probability.Martingale})
\end{itemize}

\paragraph{Other proof assistants.}
Probability theory has been formalized in various proof assistants:
\begin{itemize}
\item \textbf{Isabelle/HOL}: The Archive of Formal Proofs contains probability theory, including martingale convergence theorems.
\item \textbf{Coq}: The Mathematical Components library and Coquelicot provide analysis foundations. The Infotheo library addresses information theory.
\item \textbf{Mizar}: Contains extensive measure theory.
\end{itemize}

We are not aware of prior formalizations of de Finetti's theorem in any proof assistant.

\subsection{Multiple Proofs in Formalizations}

Formalizing multiple proofs of the same theorem is relatively rare. Notable examples include:
\begin{itemize}
\item \textbf{Fundamental theorem of algebra}: Multiple proofs exist in various proof assistants (algebraic, topological, analytic).
\item \textbf{Prime number theorem}: Different approaches have been formalized.
\item \textbf{Quadratic reciprocity}: Several proofs exist in Lean and other systems.
\end{itemize}

Our work contributes to this tradition by formalizing three proofs that use substantially different mathematical machinery (probabilistic, analytic, dynamical).

\subsection{Bayesian Foundations}

De Finetti's theorem is foundational for Bayesian statistics. The theorem justifies the Bayesian framework: if we believe observations are exchangeable (a symmetry assumption), then there exists a ``parameter'' (the directing measure) such that observations are conditionally i.i.d.\ given this parameter.

Formalizations of Bayesian reasoning include:
\begin{itemize}
\item Work on probabilistic programming semantics.
\item Formalization of conditional probability and Bayes' theorem.
\end{itemize}

Our formalization provides the measure-theoretic foundation for such developments.

\subsection{Ergodic Theory in Proof Assistants}

The Koopman proof route required developing ergodic theory infrastructure:
\begin{itemize}
\item Koopman operators and their properties
\item The Mean Ergodic Theorem
\item Shift-invariant $\sigma$-algebras
\end{itemize}

This infrastructure may be useful for future formalizations in ergodic theory, such as:
\begin{itemize}
\item Birkhoff's Ergodic Theorem (pointwise version)
\item Ergodic decomposition
\item Mixing and weak mixing
\end{itemize}

\subsection{Exchangeability Beyond Sequences}

De Finetti's theorem has been extended to:
\begin{itemize}
\item \textbf{Partial exchangeability}: Sequences with group-indexed symmetry.
\item \textbf{Arrays}: Exchangeable arrays (Aldous--Hoover theorem).
\item \textbf{Graphs}: Exchangeable random graphs (graph limits).
\end{itemize}

These extensions are potential targets for future formalization work, building on the infrastructure developed here.
